\documentclass[10pt,a4paper]{article}
\usepackage[utf8]{inputenc}
\usepackage[english]{babel}
\usepackage[T1]{fontenc}
\usepackage{amsmath}
\usepackage{amsfonts}
\usepackage{amssymb}
\usepackage{subcaption}
\usepackage{makeidx}
\usepackage{graphicx}
\usepackage{fourier}
\usepackage{listings}
\usepackage{color}
\usepackage{hyperref}
\usepackage[left=2cm,right=2cm,top=2cm,bottom=2cm]{geometry}
\author{Sigve Harang, Lukas Powalla}
\title{Space Physics Project , FYS3610}
\graphicspath{{./data/}} %BEEEEE CAREFUL WITH THIS LINE
\lstset{language=C++,
	keywordstyle=\bfseries\color{blue},
	commentstyle=\itshape\color{red},
	stringstyle=\color{green},
	identifierstyle=\bfseries,
	frame=single}
\begin{document}
\maketitle
\newpage
\tableofcontents
\newpage
\section*{Introduction}
In this project we are going to analyse the ionospheric and auroral dynamics and its relation to the solar wind conditions. Furthermore, we will compare our observations to theory. We will investigate in a specific timeslot of 6 hours from the 6. January 2011 from 18:00 o'clock to the 07. January 2011 00:00 o'clock.   
\section{Theory part}
In order to understand the movements and changes in the system of the earth and the sun, we want to introduce some basic concepts of space physics. 
A really important steady state picture of what is going on with respect to the earth's magnetic field, the interaction of the earth's magnetic field with the IMF and the consequences of the earth's environment (such as current flows or aurora) is the Dungey cycle. (the Dungey cycle is only valid when the IMF is southward orientated)
The Dungey cycle is as already mentioned a steady state picture. This means that the magnetic flux added to the open magnetic field lines through day side reconnection is equal to the magnetic flux subtracting from the open magnetic field lines through night side reconnection.
However, this cycle is in reality not a steady state picture. We observe different phases such as adding magnetic flux (growing of the polar cap), some kind of explosive expansion (subtraction of magnetic flux from the polar cap) and a recovery phase. This phenomena is called a "Auroral substorm". In addition to that, we can extract information about what is going on with the system earth-sun by looking at different data sets containing information about the magnetic field on earth, the solar wind bulk speed, the auroral intensity, the current density and further data from different physical measurements. 
Based on this measurements we will interpret the received data and relate it to the theoretical predictions and the models we introduced in the theoretical part.

\subsection{The Dungey cycle - a steady state picture \label{Dungey cycle}}

\begin{figure}[h]
\centering
\caption{Dungey cycle}
\label{aurora substorm}
\includegraphics[scale=0.5]{solvind.jpg}
\end{figure}

If the magnetic field is northward orientated ($B_z>0$) nothing happens at the boundary of the terrestrial magnetic field and the magnetic field of the sun/ the solar wind. IMF can not mix with the terrestrial magnetic field because everything in frozen in. Since the terrestrial magnetic field and the IMF are not anti-parallel orientated, magnetic reconnection does not occur and hence, everything is and stays frozen in. 
However, if the IMF is southward orientated ($B_z<0$), we do have anti-parallel orientated magnetic field at the boundary of terrestrial magnetic field and the IMF (at the magnetopause). Therefore, magnetic reconnection occurs at the magnetopause, which causes a cyclic reconnection pattern at day/night-side. If we assume a steady state picture, we call the cycle, which happens for southward IMF, the dungey cycle. The different parts of the cycle can be described as follows:
\begin{itemize}
\item[1] opposite polarity field lines reconnect at magnetopause (if IMF southward orientated)
\item[2] magnetic field lines are still connected to the solar wind. (frozen in again) Therefore, the magnetic field line is draged along with the solar wind in the direction of the tale. (magnetic tension force)
\item[3] magnetic flux is added to the tail and compresses the plasma sheet (magnetic pressure increases, plasma sheet gets thinner and thinner)
\item[4] magnetic reconnection occurs in the tail. 
\item[5] reconnected magnetic field lines return to the day side (imagination of "rubber bands" help to understand this)
\end{itemize}

\subsection{Auroral substorm}
As already mentioned in the introduction, the assumption of a steady state picture is not always a appropriate description of what happens in reality. To specify this, the rate of adding magnetic flux and the rate of subtracting magnetic flux from the polar cap are in general not equal. However, after studying data from different measurements, Akasofu could state out in 1964 different phases of a cycle so called:
\begin{itemize}
\item[1] groth phase
\item[2] expansion phase
\item[3] recovery phase
\end{itemize}
\subsubsection{groth phase}
The day side reconnection dominates and the aurora moves to lower latitudes. The auroral activity increases in the cusp and open magnetic flux is added to the polar cap. The magnetic flux, which is added due to the day side reconnection makes the tail radius increase.(because the field lines are draged tail-wards with the IMF )
\subsubsection{Expansion phase}
Sudden onset of night-side reconnection at the NENL. Sudden brightening of the night side aurora. Poleward and westward movement of the aurora (on average). It closes open magnetic flux. 
\subsubsection{recovery phase}
The night side reconnection continues at smaller rates and the auroral display quietens. Furthermore, the polar cap contracts and the magnetosphere returns to pro groth phase "equilibrium". 
\begin{figure}[h]
\centering
\caption{Auroral substorm}
\label{aurora substorm}
\includegraphics[scale=0.5]{solvind2.jpg}
\end{figure}

\subsection{Currents}

When we are dealing with the interaction of the earth's magnetic field with the IMF, it seems to be reasonable to investigate into how the currents close to earth behave. 
The reason for this is due to the connection of magnetic fields and flowing currents. We know from Maxwell's equations that flowing electric currents cause a magnetic field. In particular,this relation is given by Ampere's law (which follows from Maxwell's equations):
\begin{align}
\nabla \times \vec{B}= \mu_0 \cdot \vec{J}
\end{align} 
Therefore, we can derive the current flow by looking at the magnetic field. (in particular by looking at the curl of the magnetic field)(see measurement AMPERE)

In general, we can express the current flow in a given system with given magnetic field and electric field as follows:
\begin{align}
\vec{j}&= e n ( \vec{v}_i - \vec{v}_e ) = \sigma_p \vec{E}'_{\perp} + \sigma_H \frac{\vec{E}'_{\perp} \times \vec{B}}{B} + \sigma_{\parallel} \vec{E}_{\parallel}
\end{align}
Here, $\sigma_p$ is the Pedersen conductivity, $\sigma_H$ is the Hall conductivity and $\sigma_{\parallel}$ is a conductivity, which belongs to a current parallel to the electric field. We notice that the different currents caused by different conductivities flow in orthogonal directions with respect to each other. The orientation of $\vec{E}$ means parallel or orthogonal to the magnetic field. A discrete expression of the conductivities will not be needed in this report. 
\begin{align}
\vec{E}&= \vec{E}_{\perp} +\vec{E}_{\parallel}\\
\vec{E}'_{\perp}&=\vec{E}_{\perp}+ \vec{v}_n \times \vec{B}\\
\vec{E}'_{\perp}& \approx \vec{E}_{\perp}
\end{align}

\subsubsection{Earth's current system}

We want to give a short overview about the current system of the earth. We want to limit us on currents, which can actually be measured by the measurements used in this report. 
The currents, which flow near the polar cap can be described as follows:
On the polar cap, we find Hall currents, Pedersen currents and Field-aligned currents. 
The Electron Hall currents flow at lower altitudes than the Pedersen currents. The flow of the Hall currents is also associated with the so called twin cell convection.
The ionospheric Pedersen currents are connected to the Field aligned currents (FAC). The FAC are on the one hand side closed with the magnetopause current and on the other hand side with the ring current.
This field aligned currents are discovered by a Norwegian explorer called Kristian Birkeland in 1908. All current systems need to be closed, which is the case. 
The current flow on the polar cap can be seen in figure \ref{electric currents on the polar cap}.

\begin{figure}[h]
\centering
\includegraphics[width=0.5\textwidth]{polar_electric_currents.jpg}
\caption{electric currents on the polar cap, source: \url{http://www.windows2universe.org/earth/Magnetosphere/tour/tour_earth_magnetosphere_07.html&edu=high}}
\label{electric currents on the polar cap}
\end{figure}

\subsection{Twin cell convection}
If we look at the Dungey cycle ($B_z<0$), we can figure out the movement of the open/closed field lines at the polar caps, which gives us the movement of the particles at the polar cap. 
Let us first explain, what leads to the so called "Twin-cell"-convection. If we have southward IMF, the orientation of the IMF and the earth's magnetic field are opposite to each other. Like discussed in chapter \ref{Dungey cycle}, magnetic reconnection on the day side of the earth can occur. The magnetic reconnection at the day side adds open magnetic field lines to the polar cap. In other words, day side reconnection adds magnetic flux to the polar cap. 
The deformation of the usually round open/closed-field boundary leads to a movement, which tries to "smooth out" the little bump in the boundary. Magnetic field lines naturally want themselves to orientate in a straight line. 

Let us now look at the night side. If we look at the night side events, we know that magnetic reconnection occurs in the tail.(dungey cycle) Reasons for this happening are for example that through day side reconnection, magnetic flux is added to the tail, the magnetic pressure increases and leads in the end to magnetic reconnection in the tail. 
The magnetic reconnection to the tail closes open magnetic field lines. This means that the polar cap shrinks a bit on the night side. In other words, magnetic flux is subtracted at the night side of the earth from the polar cap. The subtraction leads to a little dint in the polar cap. The magnetic field lines tend to "smooth out" the little dint, which leads to a corresponding movement. 

In total, we expect now a movement, which is well known as the twin cell convection. The electric potential on the poles will give us the movement of the charged particles. The currents move along potential lines. The equipotential lines are therefore the trajectories of the charged particles. The currents are called "Hall"-currents. The hall-currents can be found in an altitude of approximately 110 km. (electro-jets)


\section{Observations}

All in all, we use five different sources and data to draw an image of what is going on with the interaction of the IMF with the earth's magnetic field. We will just shorty list the used methods in this reports. A detailed description of each of the methods can be found in the separated chapters. 
First of all, we use a satellite called ACE (Advanced composition explorer) in order to get knowledge about the magnetic field of the IMF at earth's position, which tells us whether day side magnetic reconnection occurs or not . 
We also use Groundbasedmagnetometer to get data from the magnetic field on earth. Depressions in the magnetic field can finally be used to detect auroral sub storms. 
Moreover, we want to look at data from SuperDARN in order to know the movement of charged particles in the ionosphere.
The data from AMPERE gives rise to Region 1 and Region 2 currents. 
Finally, the All-sky-Camera at Svalbard helps us to detect sub storms.  


\subsection{ACE}
In this section, we want to discuss the data from the ACE. (Advanced Composition Explorer)
The ACE satellite orbits the Lagrangian point $L_1$.
\begin{figure}[h]
\centering
\includegraphics[scale=0.06]{ACEposition.jpg}
\caption{position of ACE, source: \url{http://www.nasa.gov/mission_pages/sunearth/news/ace-15th.html} }
\label{position of ACE}
\end{figure}

This Lagrangian point sits exactly between the earth and the sun.(see figure \ref{position of ACE}) The solar wind particles move radially outwards from the sun. Therefore, the data from ACE can be used to describe the IMF at earth's position when we take the time into account, which is needed from the solar wind to travel from the position of the ACE to earth. 
If we take a look at the position of ACE and the solar-wind speed at the position of ACE, we can calculate the time the IMF needs to travel from ACE to earth. 

\begin{figure}[h]
\centering
\begin{subfigure}{0.45\textwidth}
\centering
	\includegraphics[width=\textwidth]{ACE_solarwindspeed.pdf}
	\caption{ Solarwind data\label{ACE Solarwindspeed}}
\end{subfigure}
\begin{subfigure}{0.45\textwidth}
\centering
\includegraphics[width=\textwidth]{ACE_magneticfield.pdf}
\caption{magnetic field measured at the ACE}
\end{subfigure}
\caption{ACE data}
\end{figure}

\begin{figure}[h]
\centering
\includegraphics[width=0.5\textwidth]{ACE_distance.pdf}
\caption{ ACE distance from earth \label{ACE distance}}
\end{figure}
Finally, we know what the IMF looks like at earth's position at a given time on earth. 
This is important because we know from the chapter \ref{Dungey cycle}, that the $B_z$ tells us something about magnetic reconnection on the day side. If $B_z>0$ (or northward IMF orientation), there is no magnetic reconnection on the day side. There is no open magnetic flux added to the polar cap. 
However, if the $Bz$ component is negative (or southward orientation of IMF) magnetic reconnection on the day side can occur and the dungey cycle kicks in. 
It is therefore very important to get data from the $B_z$ component of the IMF. The other components of the IMF can be used at a later point to tell something about the specific shape of the twin-cell convection. In figures \ref{ACE Solarwindspeed} and \ref{ACE distance}, we find data from the solarwindspeed and the ACE position. Both together will be used to determine the time delay of the IMF. The coordinate system GSE (Geocentric Solar Ecliptic). If we assume a average velocity of the solar wind (see \ref{ACE Solarwindspeed} $V_x$) of $400 \frac{m}{s}$ and a average distance of $1.4802 \cdot 10^{6} \mathrm{km}$ from ACE to earth, we get a average time delay of 1 hour and.(range between 55 minutes and 70 minutes) This is the time, the solar wind needs to travel from ACE to earth. 


\clearpage
\subsection{Groundbasedmagnetometer}

The Groundbasedmagnetometer we use is a collection of 32 station on the surface of the earth measuring the magnetic field. (by the use of magnetometers) 
The collection is called IMAGE (International Monitor for Auroral Geomagnetic Effects). We have chosen a night side event, which means that we have chosen a time when the All-sky-camera in Svalbard in on the night side. 
If we have a look at our data, we should be aware of indications concerning auroral sub storms. During a sub storm, the westward electro-jets increase. The currents cause according to Ampere's rule a depression in the north-south component of the magnetic field on earth. After the depression, the magnitude of the north-south component slowly increases to its original value.
Observing this depression is therefore a first hint of a sub storm. 
Let us now have a closer look at the data. 

\begin{figure}[h]
\centering
\begin{subfigure}{0.3\textwidth}
\centering
	\includegraphics[width=\textwidth]{X_gram.jpg}
	\caption{ Groundbasedmagnetometer x-direction \label{GBM_X}}
\end{subfigure}
\begin{subfigure}{0.3\textwidth}
\centering
	\includegraphics[width=\textwidth]{Y_gram.jpg}
	\caption{ Groundbasedmagnetometer y-direction \label{GBM_Y}}
\end{subfigure}
\begin{subfigure}{0.3\textwidth}
\centering
	\includegraphics[width=\textwidth]{Z_gram.jpg}
	\caption{  Groundbasedmagnetometer z-direction\label{GBM_Z}}
\end{subfigure}
\caption{Data from Groundbasedmagnetometer, different components }
\label{GBM_all in all}
\end{figure}

In figure \ref{GBM_all in all}, you can see data from IMAGE. The figures \ref{GBM_X}, \ref{GBM_Y} and \ref{GBM_Z} show different orthogonal components of the magnetic field. In figure \ref{GBM_X}, you can see the south- north component of the magnetic field, which is the important component for us. The vertical axis of the plots show the time and the different horizontal lines are the magnetic field components measured at different stations on earth. On the right side, you can see abbreviations of the different stations. Note, that they are ordered concerning their latitude. 
The first thing, which strikes the eye is that the magnetic field component is mostly constant from 18:00 o'clock until approximately 22:00 o'clock. At approximately 22:00 o'clock (depends on the latitude of the station), we can see fluctuation of the magnetic field component and then a sudden depression of the magnetic field component. 
This is a strong hint to a sub storm. We can also see a poleward moving of the depression especially for the 8 most northward stations (KEV to NAL). We can also see that the stations MEK to TAR don't measure a depression of the magnetic field. The stations KEV to OUJ measure the depression approximately at the same time. 

\clearpage

\subsection{SuperDARN}

SuperDARN stands for  Super Dual Auroral Radar Network and this network consists of more than 30 low-power HF radars that look into Earth's upper atmosphere. (from mid-latitudes to polar regions) The radars can among others be used to measure the motion of charged particles in the ionosphere via the Doppler shift effect. (scattering of radiation)
The motion of the plasma is interesting for us because we know what we expect from the plasma movement. In the ionosphere at approximately 110 km altitude, we expect to see the twin cell convection if the orientation of the IMF is southward. The twin cell convection is therefore related to the dungey cycle. 
How can we visualize the motion of the hall currents? Unfortunately, we have not a complete data coverage all over the polar cap. Therefore, it seems to be useful to search for another way to visualize the motion of the plasma. 
The velocity of the hall currents is: $\vec{v}=\frac{\vec{E}\times \vec{B}}{B^2}$ and the formula for the electric potential is: $\vec{E}=-\vec{\nabla} \phi$. It now tuns out that lines of equal potential are perpendicular to the electric field, which is also the case for the hall currents. It can be shown that indeed the electric potential can describe the motion of the hall currents. The currents move then on equal potential lines. 
We use the data of the Radars (motion of plasma) to make a expansion in spherical harmonics for the electric potential. The plotting of the potential can then be used to describe the plasma flow. 

\begin{figure}[h]
\centering
\begin{subfigure}{0.3\textwidth}
\centering
	\includegraphics[width=\textwidth]{Superdarn1.jpg}
	\caption{ Superdarn at 18:00 o'clock \label{Super_18}}
\end{subfigure}
\begin{subfigure}{0.3\textwidth}
\centering
	\includegraphics[width=\textwidth]{Superdarn2.jpg}
	\caption{ Superdarn at 19:00 o'clock \label{Super_19}}
\end{subfigure}
\begin{subfigure}{0.3\textwidth}
\centering
	\includegraphics[width=\textwidth]{Superdarn3.jpg}
	\caption{ Superdarn at 20:00 o'clock \label{Super_20}}
\end{subfigure}
\begin{subfigure}{0.3\textwidth}
\centering
	\includegraphics[width=\textwidth]{Superdarn4.jpg}
	\caption{ Superdarn at 21:00 o'clock \label{Super_21}}
\end{subfigure}
\begin{subfigure}{0.3\textwidth}
\centering
	\includegraphics[width=\textwidth]{Superdarn5.jpg}
	\caption{ Superdarn at 22:00 o'clock \label{Super_22}}
\end{subfigure}
\begin{subfigure}{0.3\textwidth}
\centering
	\includegraphics[width=\textwidth]{Superdarn6.jpg}
	\caption{ Superdarn at 23:00 o'clock \label{Super_23}}
\end{subfigure}
\begin{subfigure}{0.3\textwidth}
\centering
	\includegraphics[width=\textwidth]{Superdarn7.jpg}
	\caption{ Superdarn at 00:00 o'clock, next day \label{Super_00}}
\end{subfigure}
\begin{subfigure}{0.3\textwidth}
\centering
	\includegraphics[width=\textwidth]{Superdarn8.jpg}
	\caption{ Superdarn at 01:00 o'clock \label{Super_01}}
\end{subfigure}
\begin{subfigure}{0.3\textwidth}
\centering
	\includegraphics[width=\textwidth]{Superdarn14.jpg}
	\caption{ Superdarn at 07:00 o'clock \label{Super_07}}
\end{subfigure}
\caption{data from Superdarn for different times}
\label{Super_overview}
\end{figure}
The plot of the electric field on the polar cap is shown in figure \ref{Super_overview} for different times. 
First of all we notice, that the shape of the electric potential fits to the theoretical predictions from theory. We can see two cells, which can be characterised through a region with negative electric potential (blue) and a region with positive electric potential (red). In the region with negative electric potential, the charged particles move clockwise and in the positive region the particles move anti-clockwise. 
Also the orientation of the two cells fits. The direction to the sun is in the figures \ref{Super_overview} at 12 o'clock. (referring to a watch for defining direction with 12 o'clock on top) The twin cells should be separated by a line from approximately 6 o'clock to 12 o 'clock. This depends also on the components of the IMF. The orientation of the twin cells seems to be right.   All in all, we can say that the theory matches the experimental data. 
We can see that the size of the polar cap (which is defined as the region, where we have open magnetic flux) variates with time as well as the size of the twin ell convection. From 18:00 o'clock (see figure \ref{Super_18}) to approximately 22:00 o'clock, we can see that the twin cell stays at hight latitude. The polar cap seems to be small. The magnetic flux through the polar cap variates from $\phi_{pc}=31 k V$ to $\phi_{pc}=41$. 

\subsection{AMPERE}
AMPERE (Active Magnetosphere and Planetary Electrodynamics Response Experiment) is a earth observing system, which can be used to measure near-realtime magnetic field by 66 commercial satellites.
The data from the magnetic field collected by AMPERE can be used to improve our understanding of space weather.
In our project, we want to use AMPERE to determine the Birkeland currents. The Birkeland currents are field aligned currents. As described in the theory part, flowing electric currents cause a curl of the magnetic field. If there are such FAC, we should also see a curl in the magnetic field measured by the satellites.
We don't want to have a look at the explicit magnetic field data, but directly at the simulated current flow based on the measured magnetic field of AMPERE. This means that we expect to see the Region 1 currents at high latitudes and the Region 2 currents at lower latitude. 
In addition to that, we are interested to see how the currents change with respect to time. 


\begin{figure}[h]
\centering
\begin{subfigure}{0.3\textwidth}
\centering
	\includegraphics[width=\textwidth]{ampere0.png}
	\caption{ 18:00 o'clock\label{amp18}}
\end{subfigure}
\begin{subfigure}{0.3\textwidth}
\centering
	\includegraphics[width=\textwidth]{ampere1.png}
	\caption{19:00 o'clock \label{amp19}}
\end{subfigure}
\begin{subfigure}{0.3\textwidth}
\centering
	\includegraphics[width=\textwidth]{ampere2.png}
	\caption{ 20:00 o'clock \label{amp20}}
\end{subfigure}
\begin{subfigure}{0.3\textwidth}
\centering
	\includegraphics[width=\textwidth]{ampere3.png}
	\caption{ 21:00 o'clock \label{amp21}}
\end{subfigure}
\begin{subfigure}{0.3\textwidth}
\centering
	\includegraphics[width=\textwidth]{ampere4.png}
	\caption{ 22:00 o'clock \label{amp22}}
\end{subfigure}
\begin{subfigure}{0.3\textwidth}
\centering
	\includegraphics[width=\textwidth]{ampere5.png}
	\caption{ 23:00 o'clock \label{amp23}}
\end{subfigure}
\begin{subfigure}{0.3\textwidth}
\centering
	\includegraphics[width=\textwidth]{ampere6.png}
	\caption{ 00:00 o'clock (next day) \label{amp00}}
\end{subfigure}
\begin{subfigure}{0.3\textwidth}
\centering
	\includegraphics[width=\textwidth]{ampere7.png}
	\caption{ 01:00 o'clock \label{amp01}}
\end{subfigure}
\begin{subfigure}{0.3\textwidth}
\centering
	\includegraphics[width=\textwidth]{ampere8.png}
	\caption{ 02:00 o'clock \label{amp02}}
\end{subfigure}
\caption{Data from AMPERE at different times}
\end{figure}



\subsection{Svalbardimager}

\begin{figure}[h]
\centering
\begin{subfigure}{0.45\textwidth}
\centering
	\includegraphics[width=\textwidth]{SvalbardImager5577A.png}
	\caption{ SvalbardImager for 5577 $\r{A}$ \label{SBI_5_overview}}
\end{subfigure}
\begin{subfigure}{0.45\textwidth}
\centering
	\includegraphics[width=\textwidth]{SvalbardImager6300A.png}
	\caption{ SvalbardImager for 6300 $\r{A}$\label{SBI_6_overview}}
\end{subfigure}
\caption{Overview over the Svalbard-Imagers}
\end{figure}

\begin{figure}[h]
\centering
\begin{subfigure}{0.3\textwidth}
\centering
	\includegraphics[width=\textwidth]{SvalbardImager5577A18.png}
	\caption{ SvalbardImager at 18:00 o'clock \label{SBI_5_18}}
\end{subfigure}
\begin{subfigure}{0.3\textwidth}
\centering
	\includegraphics[width=\textwidth]{SvalbardImager5577A19.png}
	\caption{ SvalbardImager at 19:00 o'clock \label{SBI_5_19}}
\end{subfigure}
\begin{subfigure}{0.3\textwidth}
\centering
	\includegraphics[width=\textwidth]{SvalbardImager5577A20.png}
	\caption{ SvalbardImager at 20:00 o'clock \label{SBI_5_20}}
\end{subfigure}
\begin{subfigure}{0.3\textwidth}
\centering
	\includegraphics[width=\textwidth]{SvalbardImager5577A21.png}
	\caption{ SvalbardImager at 21:00 o'clock \label{SBI_5_21}}
\end{subfigure}
\begin{subfigure}{0.3\textwidth}
\centering
	\includegraphics[width=\textwidth]{SvalbardImager5577A22.png}
	\caption{ SvalbardImager at 22:00 o'clock \label{SBI_5_22}}
\end{subfigure}
\begin{subfigure}{0.3\textwidth}
\centering
	\includegraphics[width=\textwidth]{SvalbardImager5577A23.png}
	\caption{ SvalbardImager at 23:00 o'clock \label{SBI_5_23}}
\end{subfigure}
\caption{Ketograms from the SvalbardImager for $\lambda=5577 \cdot 10^{-10} \mathrm{m}$, different times }
\end{figure}

\begin{figure}[h]
\centering
\begin{subfigure}{0.3\textwidth}
\centering
	\includegraphics[width=\textwidth]{SvalbardImager6300A18.png}
	\caption{ SvalbardImager at 18:00 o'clock \label{SBI_6_18}}
\end{subfigure}
\begin{subfigure}{0.3\textwidth}
\centering
	\includegraphics[width=\textwidth]{SvalbardImager6300A19.png}
	\caption{ SvalbardImager at 19:00 o'clock \label{SBI_6_19}}
\end{subfigure}
\begin{subfigure}{0.3\textwidth}
\centering
	\includegraphics[width=\textwidth]{SvalbardImager6300A20.png}
	\caption{ SvalbardImager at 20:00 o'clock \label{SBI_6_20}}
\end{subfigure}
\begin{subfigure}{0.3\textwidth}
\centering
	\includegraphics[width=\textwidth]{SvalbardImager6300A21.png}
	\caption{ SvalbardImager at 21:00 o'clock \label{SBI_6_21}}
\end{subfigure}
\begin{subfigure}{0.3\textwidth}
\centering
	\includegraphics[width=\textwidth]{SvalbardImager6300A22.png}
	\caption{ SvalbardImager at 22:00 o'clock \label{SBI_6_22}}
\end{subfigure}
\begin{subfigure}{0.3\textwidth}
\centering
	\includegraphics[width=\textwidth]{SvalbardImager6300A23.png}
	\caption{ SvalbardImager at 23:00 o'clock \label{SBI_6_23}}
\end{subfigure}
\caption{Ketograms from the SvalbardImager for $\lambda=6300 \cdot 10^{-10} \mathrm{m}$, different times }
\end{figure}


\clearpage
\section{discussion}

\section{conclusion}


 

\end{document}